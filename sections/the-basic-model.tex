
\section{The Basic Model}

\textbf{Keywords} community, member, registry, decision, event, domain, influence, proposal, vote, ballot, resolution, resolution outcome.

First let us describe informally what it means to make collaborative decision-making. First a notion of \textbf{community} as a set of \textbf{members} is required, all of the decision power is encapsulated within such community. Secondly, the community must have a \textbf{registry} to authenticate and authorise members. Following, \textbf{decisions} may be triggered by incoming \textbf{events} which are categorised by \textbf{domain}. Then a separation is made between the \textbf{influence} that each member has on the decision, and the way in which a decision is \textbf{resolved} into a \textbf{resolution outcome} after \textbf{votes} over \textbf{proposals} have been collected in a \textbf{ballot}, the ballot represents not vote counts, but accumulated influence of the voting members on a proposal.