\section{Introduction}

Kratia is an abstract model, and a Kratia Engine is any software that implements the model, it enables communities and organisations with digital governance. It offers alternatives to the decision-making process, first, options to the way influence is distributed to the members, and second, options to the way a decision is resolved once members have voted for proposals. It helps the communities grow and evolve faster and healthier, because besides automating the collaborative decision making, it also uses the same system to let the community change its current governance method.
There are several exciting industries which would benefit: open software projects, blockchain networks, digital nomad groups, small and big companies, NGOs, distributed autonomous organisations, governments. Also exciting possible features: webhooks for automation, automatic transparency, sub-communities for scaling, composition and combination of complex decision systems, open data for research on governance systems.

The model is not just words, in this document we will try to completely boil down every important part to a formal model in order to provide denotational semantics to the Kratia Engines, to find provable truths about the model, and to have a real source of documentation.

\textbf{Disclaimer} this is a living document, until a final version is done, everything (minus probably the base model) is subject to change. Comments and contributions are welcome, please use one of the contact methods to do so, GitHub issues are probably the best for such endeavour.